% !TeX spellcheck = en_US
% -*-coding: utf-8 -*-

%% Conceptos
%% ---------

\section{Conceptos de Ingeniería de Lenguajes basada en Modelos}
\label{sec:conceptos}

La MDLE busca aplicar los conceptos de MDE para la creación de lenguajes.
Esto supone la confluencia de terminos y conceptos tanto de desarrollo basado en modelos como de las tecnologías tradicionales de análizadores sintácticos y compiladores.
Esta sección presenta algunos de esos conceptos.

\subsection{Ingeniería basada en Modelos}

En Ingeniería de Software basada en Modelos (MDSE), desarrolladores definen la funcionalidad del sistema usando modelos, 
usan operaciones sobre modelos para hacer simulaciones y verificar las especificaciones, 
y finalmente usan transformaciones de modelos para generar el código fuente de las aplicaciones.

Usando una propuesta de MDSE como la \textit{Model Driven Architecture} (MDA) de la \textit{Object Management Group} (OMG), 
los desarrolladores construyen un modelo independientes de la plataforma (PIM, \textit{Platform Independent Model})
usando un  lenguaje de dominio específico (DSL, \textit{Domain Specific Language}).
El PIM es entonces transformado en uno o más modelos específicos de plataformas (PSMs, \textit{Platform Specific Models}) como .Net o Java.
Finalmente, estos modelos específicos de plataforma son transformados al código fuente de las aplicaciones. 
Por ejemplo un analista de negocios puede crear una especificación para un proceso de negocio una interaccion del usuario y luego generar el código fuente correspondiente usando transformaciones de modelos.
Sistemas de AndroMDA y OpenMDX pueden ser usados para desarrollar aplicaciones basadas en MDA.

Además de MDA, existen muchas aproximaciones basadas en modelos para la creación de aplicaciones.
Por ejemplo, HEDL permite crear bases de datos y código java JPA a partir de un DSL que define las entidades.
UWE, OOH4RIA y WebML permiten crear aplicaciones web completas a partir de varios modelos describiendo requerimientos de datos, de navegación, de presentación y de funcionalidades.
Applause y Moble permiten crear aplicaciones móviles que funcionan en Android y iPhone a partir de un DSL que define los formularios y la base de datos de la aplicación. 
Así mismo, Gaspard2 y \ldots permiten crear software para sistemas embebidos usando modelos que representan la funcionalidad y los bloques funcionales y de hardware de la solución.

\subsubsection{Meta-Modelos y Modelos}

La MDSE se centra en el uso de modelos y meta-modelos.
Un modelo es una representación abstracta del algún dominio relacionado con el problema o con una solución informática.
Estos modelos son ampliamente utilizados en ingeniería de software.
Por ejemplo, un desarrollador puede hacer un modelo entidad relación, un modelo de clases o un modelo de actividades para representar los requerimientos y/o el diseño de una aplicación de software.
La figura \ldots muestra un modelo entidad relación de un \ldots  

Un meta-modelo es una especificación de algún tipo de modelo. 
Este metamodelo define los elementos que se pueden colocar en un modelo, así como algunas reglas que estos elementos deben seguir.
Por ejemplo, un modelo entidad relación se compone de entidades, atributos y relaciones, y no es posible colocar un atributo por fuera de una entidad.
La figura \ldots muestra una definición del modelo entidad relación usando un modelo de meta-clases.

Cuando un modelo está definido de acuerdo a los elementos y reglas de un metamodelo se dice que el modelo \textit{es conforme} con ese metamodelo.
Por ejemplo, el modelo mostrado en la figura \ldots es conforme con el metamodelo en la figura \ldots 

En la actualidad, los metamodelos usualmente se definen usando uno de dos estándares: MOF o ECore.
MOF (\textit{Meta-Object Facility}) es el estándar de metamodelamiento definido por la OMG para crear los lenguajes de metamodelamiento UML y BPMN entre otros.
Ecore es una versión simplificada de MOF creada por IBM y que hace parte de Eclipse Modeling Framework.
En la actualidad, Ecore es ampliamente usado en infinidad de lenguajes, compiladores y editores que funcionan sobre la plataforma Eclipse. 

\subsubsection{Semántica estática}

En muchas ocasiones, la sola construcción de un metamodelo basado en meta-clases.
En esos casos, al igual que con los diagramas de UML, es posible definir reglas usando condiciones y restricciones en OCL.

Por ejemplo, en el modelo entidad-relación no es posible que exista una entidad sin un nombre.
Esta regla no se puede colocar en el modelo de metaclases sin indicar una restricción.
Figura \ldots muestra una invariante en OCL que establece que no puede existir una entidad sin nombre.

Se conoce como Semática Estática o Reglas de Bien Formado (WFR, \textit{Well formed rules}) al conjunto de reglas que permiten saber si un modelo está correctamente construido.
En algunos casos, estas reglas pueden clasificarse para determinar si corresponden a errores en el modelo o solo a advertencias. 


\subsubsection{Sintáxis concreta}

En MDSE los modelos pueden ser visualizados de diferentes formas.
Por ejemplo, un programa puede representarse en el código fuente, un flujograma o un diagrama de actividades en UML.
Aunque todas estas tienen una apariencia diferente, todas ellas son representaciones del mismo modelo.

Una sintáxis concreta es la forma textual o gráfica como un modelo en particular puede ser representado.
Normalmente los entornos de desarrollo, como Eclipse, permiten tener varias sintáxis concretas para el mismo modelo.
Figura \ldots muestra tres posibles sintáxis concretas para el modelo presentado en la figura \ldots
Allí puede verse que el mismo modelo entidad relación puede ser representado usando un diagrama, una estructura jerárquica (en árbol) o una representación textual.


\subsubsection{Transformaciones de Modelos}

Las aproximaciones de MDSE se basan en la transformación de modelos para crear nuevos modelos, documentación y código fuente.
Existen dos tipos de transformaciones: 
(1) las transformaciones de modelo a modelo (conocidas como transformaciones M2M) 
y (2) las transformaciones de modelo a texto (conocidas como transformaciones M2T)

\ldots



\subsection{Patrones en el desarrollo de lenguajes}

De acuerdo a Terence Par, existen varios patrones identificables en la implementación de un lenguaje de programación.
En primer lugar, típicamente cualquier programa en un lenguaje particular es procesado por un analizador sintáctico que genera una representación intermedia (p.ej. un árbol de sintáxis abstracta).
Cuando el lenguaje es interpretado, el mismo u otro programa toma esa representación interna y ejecuta cada una de las instrucciones una a una de acuerdo a una reglas semánticas prestablecidas.
En el caso que el lenguaje sea compilado, el mismo u otro programa toma esa representación interna y genera el código correspondiente en un lenguaje de propósito general (p.ej. Java) o un lenguaje de bajo nivel (p.ej. assembler o código binario). 
 
(ver gráfico general de Terence Par en Language implementation patterns)
http://pragprog.com/book/tpdsl/language-implementation-patterns

Algunas herramientas para la creación de lenguajes, como Stratego/XT, permiten definir una grámatica y generar automáticamente el análizador sintáctico.
Adicionalmente, es posible definir unas reglas de re-escritura que transforman el árbol de sintáxis que genera el analizador sintáctico en código fuente en otro lenguaje.

\subsubsection{Lenguaje específico de dominio}

Un lenguaje específico de dominio (DSL) es un lenguaje que esta especializado en modelar o resolver un conjuntos especifico de problemas.

Debido a la naturaleza específica de estos lenguajes, ellos pueden representar aspectos estáticos o dinámicos de algún dominio.
Por ejemplo, SQL representa un conjunto de DSLs:
el lenguaje de definición de datos (DDL, \textit{Data Definition Language}) es un lenguaje que permite definir la estructura de la base de datos,
el lenguaje de manipulación de datos (DML, \textit{Data Manipulation Language}) permite ejecutar consultas y actualizaciones sobre los dayos,
y el lenguaje de módulos almacenados (PSM, \textit{Persistent Stored Modules}) permite definir procedimientos almacenados en la base de datos.
En este caso, mientras el lenguaje DDL permite definir aspectos estáticos (la estructura), el lenguaje PSM permite definir aspectos dinámicos (algoritmos y programas).   

\subsubsection{Sintáxis y Grámatica de Lenguaje de Programación}

En lenguajes de programación, la sintáxis es la forma correcta en que dede estar dispuestos los simbolos que componen una instrucción ejecutable por el ordenador.

La sintáxis está determinada por una grámatica del lenguaje.
Esta grámatica define las reglas que se debe usar para escribir un programa en ese lenguaje.
Por ejemplo, la grámatica de Java define como se deben escribir los programas en Java.
Estas grámaticas suelen especificarse usando algún lenguaje de definición de grámaticas. 
Normalmente los lenguajes son definidos usando BNF (\textit{Backus Naur Form}), EBNF o alguna variación.

\ldots



\subsection{Ingeniería de Lenguajes basado en Modelos}

La Ingeniería de Lenguajes basado en Modelos es la aplicación de principios de MDE para la construcción de lenguajes, interpretadores y compiladores.
Usando esta aproximación, los desarrolladores del lenguaje se centran en especificar los metamodelos, las grámaticas y las transformaciones a los lenguajes de bajo nivel.
Luego, usando estas especificaciones, los desarrolladores pueden utilizar herramientas basadas en modelos para generar los analizadores sintácticos, editores e interpretadores.

Por ejemplo, usando Xtext es posible implementar un lenguaje especificando solamente la grámatica y las transformaciones a Java.
En ese caso, un editor en Eclipse, un depurador y un compilador pueden ser generados a partir de eso.
Usando sistemas como EMfText y Kermeta, es posible definir el metamodelo, la grámatica y el comportamiento, y generar un editor, un interpretador y un depurador para el lenguaje.

Además, para implementar lenguajes más complejos, es posible usar una serie de modelos intermedios y transformaciones.
Es posible aprovechar todas las tecnologías de transformación de modelos para ejecutar optimizaciones en el código, visualizaciones, análisis estático y generación por multiples etapas.
      