% !TeX spellcheck = en_US
% -*-coding: utf-8 -*-

%% Introducción
%% ------------

\section{Introducción}
\label{sec:introduccion}

% -- contexto

La construcción de lenguajes de programación, compiladores e interpretadores ha sido un reto recurrente en computación.
Una gran variedad de aproximaciones y herramientas se han planteado para facilitar su desarrollo e implementación.
Herramientas como Lex, Yacc y Antlr han sido usadas para crear análizadores sintácticos a partir de especificaciones de la grámatica del lenguaje.
Otras herramientas, como Attribute-Grammars y JastAdd, han sido usados para proveer semántica o comportamiento a lenguajes ya definidos.
Además, soluciones como StringTemplate o Xtend han sido usados para generar código en lenguajes de bajo nivel para compilar estos lenguajes.

% -- el tema

En los últimos años, ha surgido la Ingeniería de Lenguajes basadas en modelos (MDLE, \textit{Model Driven Language Engineering}) como una alternativa para la construcción de lenguajes, interpretadores y compiladores.
Este MDE busca explotar las aproximaciones de metamodelamiento, interpretación y transformación de modelos para facilitar la creación de lenguajes. 
En la actualidad, herramientas como Xtext, EmfText y KerMeta son usados para la creación de entornos de desarrollo, lenguajes y compiladores usando conceptos de la Ingeniería basada en Modelos (MDE, \textit{Model Driven Engineering}).

% -- el documento

Este documento busca introducir al lector en la Ingeniería de Lenguajes basada en Modelos.
Igualmente busca presentar algunos de sus principales conceptos.
% metamodelos, modelos, grámaticas, sintáxis, semántica estática y dinámica.
Estos conceptos son ilustrados usando un ejemplo centrado en la creación de un lenguaje para \ldots.

% -- organización del paper

El resto del documento esta organizado de la siguiente forma:
La sección \ref{sec:conceptos} presenta los conceptos de Ingeniería de Lenguajes basada en Modelos.
La sección \ref{sec:ejemplo} ilustra tales conceptos utilizando una implementación de ejemeplo del lenguaje \ldots.
Finalmente, la sección \ldots presenta un resumen y concluye el documento.